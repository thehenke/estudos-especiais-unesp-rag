\section*{Resumo}

Em meio ao campo da IA (\textit{Artificial Intelligence}) Generativa, destaca-se a importância de encontrar um equilíbrio entre a eficiência computacional e a precisão das respostas geradas, especialmente no contexto de Retrieval-Augmented Generation (RAG). Destarte, essa monografia tem por objetivo o levantamento das abordagens e estratégias de estruturação de RAG (\textit{Retrieval Augmented Generation}). Apresentando uma análise detalhada dos benefícios e limitações de cada abordagem, o trabalho promove uma compreensão aprofundada das melhores práticas e otimizações para reduzir a latência e melhorar a precisão das respostas, proporcionando uma experiência de usuário mais eficiente e satisfatória.


\textbf{Palavras-chave:} \textit{Retrieval Augmented Generation}, \textit{Large-Language-Models},\textit{Fine Tuning}, \textit{Transformers}.